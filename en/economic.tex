% !TEX root = main.tex
\section{Econmic Model}
The cryptocurrency is endowed with economic significance, either as a kind of trading medium or intelligent asset. Therefore, a reasonable economic model can help us to establish a value measurement standard on the blockchain, which is also the objective of core nebulas rank. This chapter first introduces the mathematical representation of cryptocurrency, and then analyzes the cryptocurrency with a simple but well-recognized monetary model. During the analysis, we introduce the core nebulas rank as an important argument.

\subsection{Representation of Cryptocurrency}
The biggest difference between cryptocurrency and traditional economic entity is that all the transactions on the cryptocurrency can be traceable. This provides the data sources for us to analyze the impact of each transaction on the economic system.

In general, a cryptocurrency system can be defined as a pair $(\mathcal{L}, \mathcal{U})$, where $\mathcal{L}$ is the ledger system, and $\mathcal{U}$ is the users of cryptocurrency. Further, the ledger system can be described as a triple as below:

\begin{align}
\mathcal{L} = (\mathcal{A}, \mathcal{D}, \mathcal{T})
\end{align}

\noindent where $\mathcal{A}$ is the set of accounts, $\mathcal{D}$ is the set of initial balances of each account, and $\mathcal{T}$ is the set of transactions. Each transaction can be recorded as a tetrad as below:

\begin{align}
\mathcal{D} = \{a \rightarrow d, a{\in}\mathcal{A}, d{\in}R^*\}
\end{align}
\begin{align}
\mathcal{T} = \{(s, t, w, \tau)\}
\end{align}

\noindent where $a \rightarrow d$ represents the balance $d$ corresponding to the account $a$ ($d$ is a positive real number, in other words, we do not takes the accounts with zero balance into consideration). $s$, $t$, $w$ and $\tau$ represents the source account, target account, amount and time of a transaction respectively.

An account is controlled by a related user, who can propose a transaction with the account, which can be denoted as:

\begin{align}
u \dom a. \quad u\in \mathcal{U}, a\in \mathcal{A}
\end{align}

\noindent On one hand, a user can control multiple accounts, represented as:

\begin{align}
A(u) = \{\forall a\in \mathcal{A} : u \dom a\}
\end{align}

\noindent On the other hand, an account can only be controlled by a single user, shown as:

\begin{align}
\forall u_1, u_2 \in \mathcal{U} : A(u_1) \cap A(u_2) = \phi
\end{align}

To be notified, the model described above is a reasonable simplification of any ecryptocurrency system. In the model, we do not distinguish the on-chain data from off-chain data, do not introduce either transaction price or invocations of smart contracts and so on. In addition, the accounts of exchanges are typically specific. Generally speaking, the transactions in an exchange can be divided as two categories: normal transactions which will be recorded on the chain and intra-exchange transactions which will not be recorded in a centralized database of the exchange. This leads to that we will lose the intra-exchange transactions if we only obtain the data from the chain. 
However, if the intra-exchange transactions can be obtained with the cooperation of the exchange, we can further map an exchange account into multiple accounts, so as to use the model above.


\subsection{Model of Cryptocurrency}
Although the cryptocurrency differs largely from the traditinal commodity money and fiat money, the classical monetary theory still has the instruction value. As the current money of a new economic entity~\cite{swan2015blockchain}, cryptocurrency has the attributes of the money and three functions of money: medium of exchange, store of value, and unit of account. 

Hereby, we establish a simple and classic monetary model to help us understand the physical significance of nebulas rank.

First of all, we try to give the indicator to measure the `velocity factor' in the cryptocurrency ecosystem. 

Another concept needed to be differentiated from the `velocity factor' in the economics is `liquidity'. `Liquidity' is used to describe the difficulty level of exchanging the assets for the medium of exchange. As the money itself is a medium of exchange in the economics, the money is the assets with the best `liquidity'.

\whitepaper{In the Nebulas Technical White Paper, we used the word `liquidity'~\cite{Nabulas}. However, there is no rigid definition of `liquidity', whose meaning is very broad even in the economics. For example, the entries to explain the `liquidity' includes three totally different aspects in "The New Palgrave: A Dictionary of Economics". R. S. Kroszner pointed out that there were 2795 independent papers mentioning `liquidity' during the past 6 months, each of which raised a typically different statement though~\cite{randall}. The `liquidity' in this yellow paper is referred to as the `velocity of money', meaning the turnover times of a monetary unit within a certain period of time. }

We use the velocity of money to represent the turnover rate of cryptocurrency~\cite{selden}, namely the turnover times of a monetary unit within a certain period of time (one day in this paper), which is represented with $V$. According to the classical quantity theory of money, the equation is expressed as below:

\begin{align}
M\times V=P\times Y
\label{eq:currency}
\end{align}

\noindent where $M$, $V$, $P$ and $Y$ represent the total monetary amount of the economical system, the velocity of money, the price level (measured by the money of unit economical output, thus the money price is $\frac{1}{P}$), and real economical output (real GDP) respectively. The equation illustrates that the product of monetary amount and velocity of money equals the product of price of goods and output.

As for the monetary amount $M$, the Nebulas is similar to Ethereum that the monetary amount maintains steady growth (the additional issuance percentage of nebulas money is set as 4\% at present). Nebulas is different from Bitcoin as the total monetary amount of latter will be stable at 21 billion at last. The velocity of money $V$ can be describe as the ratio of the circulated monetary amount and the monetary supply. As a result, the equation~\ref{eq:currency} can be further expressed as:

\begin{align}
(M + \Delta{m}) \times \frac{\sum_{(s, t, w, \tau)\in \mathcal{T}}{w}}{M} = P \times Y
\label{eq:cur_ext}
\end{align}

\noindent where $\Delta{m}$ is the additional monetary supply. 

In terms of price level $P$, it is acceptable that the value of price is determined by the relationship between the monetary supply and demand, both by the classical theory of money and New Keynesian Models. In the long term, the total price level will be adjusted to make the monetary supply and demand equal.

However, the total price level cannot construct a balance between monetary supply and demand in the short term. In a healthy economical system, the growth rate of price level is usually smaller than that of velocity of money. By increasing the monetary supply (reduce the interest rate in other words), both the price level $P$ and goods/service demands will increase in the meantime. On the other side, the increase speed of price level should be controlled, to prohibit the users from holding the cryptocurrency for a long time, thus reducing the velocity. The reason for the users to hold the cryptocurrency is that they expect the price of cryptocurrency will raise.


With regard to the real economical output $Y$, it is usually represented as real GDP by economists, namely `a monetary measure of the market value of all final goods and services produced in a period of time'. We believe that the value of cryptocurrencies is based on its velocity, namely each transaction makes contributions to the total economic aggregate to a certain extent. In other words, once a transaction takes place, it both increases the velocity of cryptocurrency and people's approval/belief of cryptocurrency to some degree. As a result, we think that $Y$ in the equation~\ref{eq:cur_ext} is consisted of each transaction. Given that the subjects of a economical system are accounts, we can also explain $Y$ as the transactions issued by each account as below:

\begin{align}
Y=\sum_{a\in \mathcal{A}} \mathcal{C}(a)
\end{align}

\noindent where $\mathcal{C}(a)$ represents the contributions made by account $a$ to the economical output, namely nebulas rank.

The development of cryptocurrency relies on the development of the community. Therefore, we consider that quantifying the contribution made by each acocunt is the basis of designing the reasonal incentive mechanism. Based on this, the economical system can create either explicit incentives (e.g., Proof of Devotion in nebulas technical white paper) or implicit incentives (e.g., the sorted search results provided by search engines). 
The directive and primitive incentives in the cryptocurrency is the additional issuance of money, which is different from that in the traditional monetary theory.