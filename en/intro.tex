% !TEX root = main.tex

\section{Introduction}

Nowadays, more and more scenarios benefits from \emph{decentralization}, which
is the core of blockchain systems. For example, Bitcoin, the origin of
blockchain, has proven it's significance to digital assets, while Ethereum has
proven how important is decentralization to DApps. And there are more and more
blockchain projects explore how they can leverage decentralization.

Obviously, the backbone of decentralization in blockchain is the openness and features of anonymity.
%These features in someway secures the trading and data on
%blockchain. Consequently, no one can change users' data or get the owner of some data
%without getting users' private key.


Yet, openness and anonymity obstruct the emergence of value
measurements~\cite{meiklejohn2013fistful}. There are two aspects. First, it is
difficult to infer if some accounts belong to the same user, which means it is
difficult to build a mechanism like HTTP Cookie~\cite{Cookie}, or to use 
traditional data analysis technologies to understand user characteristics.
Second, the openness of blockchain makes it vulnerable to manipulation,
especially for value measurements. Attackers can easily get all details about the
value measurements, and figure out the weakness of the whole system. This
largely differs from traditional value measurements which are close or
independent.

We believe that the effective value measurement is the foundations of
blockchain's prosperity. Both the lack of and ineffectiveness of value measurement may confine blockchains to limited use cases.

First of all, we need a methodology to quantify the value of data,
applications and accounts on blockchains. The root cause is cooperations on
blockchain keeps scaling up, and the requirements of efficiency
keeps growing. Without value measurement, such collaboration may be negatively affected.

Second, blockchains are still at the very early stage, and the value of data
and assets on the blockchains is still underground and waiting to be found.
Effective value measurements will uncover the value and empower more applications and enable more application scenarios, for example, loans, credit, data search, personalized recommendation and cross-chain interaction.

Third, incentives, which is based on value means, is necessary to healthy blockchain ecosystems. Without effective value measurements, incentives may lead a blockchain system to corruption and eventual collapse.

As a conclusion, an effective value measurement for blockchain needs to be
\begin{itemize}
\item{\textbf{Truthful.}} The rank needs to measure some characteristic of a blockchain system, and thus can be trusted in some way;
\item{\textbf{Fair.}} This means the rank need to be manipulation-resistant, and it is the core of the rank algorithm;
\item{\textbf{Diverse.}} There will be different ranking requirements from different applications on blockchain, thus a good rank algorithm should cover different scenarios.
\end{itemize}

We believe Nebulas Rank shall be an effective value measurement for
blockchains.

For truthfulness, we define Nebulas Rank to be quantification of an account's
contribution to the blockchain system after considering many different
metrics.

We believe that cryptocurrencies should have the attributes of money, and
three functions of money: medium of exchange, store of value, and unit of
account. Blockchains themselves are economic systems and the classical monetary theory still has the instruction value. Furthermore, we believe the value of cryptocurrencies comes from the liquidity. Specifically, each transaction between users increases the liquidity of cryptocurrencies, and endows the value of cryptocurrency eventually. Thus, the on-chain transactions are effective and natural data sources for effective value measurement.


To evaluation the effectiveness of Nebulas Rank, we calculate the sum of all
accounts' Nebulas Rank on Ethereum, and compare it with the market capital
given by \texttt{coinmarketcap.com}. Our evaluation shows strong
correlation between them, about $0.84$. That means Nebulas can measure
accounts' contribution at the micro-level, while it can measure the
value of blockchain systems at the macro-level.

For justness, we involve a special function to resist manipulation, and our analysis demonstrates its performance to be manipulation-resistant.

Based on the theory of Nebulas Rank, we divide Nebulas Rank into Core Nebulas
Rank and Extended Nebulas Rank for different applications and scenarios.

Core Nebulas Rank defines the algorithm to calculate an account's contribution
to the whole blockchain system in a certain period of time. And such
calculate involves two factors: the median stake of an account in a certain
period, and the in-and-out degree of the account in a certain period.

Extended Nebulas Rank is for different applications and scenarios,
and it is based on Core Nebulas Rank. For example, we show how to rank smart
contracts based on Core Nebulas Rank; we also show how to extend Core Nebulas Rank to a multi-dimensional vector.


Besides theory and methodology of Nebulas Rank, we also present our
consideration about how to implement Nebulas Rank, including whether to 
put ranking scores on-chain, how to update the algorithm of Nebulas Rank,
and our future work on Nebulas Rank.

\whitepaper{
  Special Hint:
  The content is this yellow paper may be different from the description in our
  whitepaper (version 1.02 released on April 2018)~\cite{Nabulas}. This is because we keep
  thinking and verifying the algorithm in our whitepaper. And now we are more
  confident and capable to make it more rigorous. We use a different format
  (like this paragraph) to emphasize the relevant updates presented in this yellow paper.

}
