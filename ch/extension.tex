\section{扩展星云指数}
核心星云指数用来衡量一个账户地址对于整个加密数字货币经济总量的贡献,对于星云链规划中的贡献度证明(PoD)及开发者激励(DIP)是十分重要、且
符合其应用场景的。然而,我们也进一步注意到,有更多的场景需要不一样的价值衡量尺度,为此,针对不同的应用场景,给出了扩展星云指数。
需要注意的是,扩展星云指数基于核心星云指数,以保证在不同应用场景下,能够持续激励星云生态的发展。

\subsection{针对智能合约的扩展星云指数}
对生态中的智能合约排序,是十分重要的,一方面能够帮助用户发现更高质量的DApp,另一方面,能够激励开发出高质量应用的开发者,
使整个生态健康稳定的发展。

对智能合约的排序,基于两个事实,即账户地址对智能合约的调用,及智能合约之间的调用。我们首先将账户地址对智能合约的调用看作账户地址向智能合约分摊
自己对经济总量的贡献,从而使得每个智能合约有了初始的分值;然后将智能合约之间的调用看作有向无环图,使用Page Rank对每个智能合约计算最终的星云
指数。


\subsection{多维的扩展星云指数}
更进一步的,我们注意到一些应用需要多个维度的数据,对链上数据的相关性进行计算,例如基于区块链的广告系统,需要在多个维度对需要投放的广告及用户进行
相关性计算。在这种场景下,扩展星云指数是多维的,即,表示为一个向量,核心星云指数作为其中的一个维度。

对于多维的扩展星云指数其他维度,我们认为依赖于具体的应用场景,可以参考本文中对于核心星云指数的计算方法。对于多维的扩展星云指数的使用,同样依赖于具体
的应用场景。

\vspace{2em}

我们通过智能合约的扩展星云指数这样一个真实的应用场景,描述了扩展星云指数的一种实现方式,给出了相应的价值衡量尺度;
更进一步的,我们给出了多维的扩展星云
指数,为更多应用场景下的价值衡量尺度提供了可能性。
