% !TEX root = main.tex

\section{Nebulas Rank Core 如何抵抗操纵?}
本节分析Nebulas Rank Core应对操纵的情况。

“操纵”指的是攻击者作出特定行动以获得最高的利益。攻击者的行动空间是利用自己和盟友所能控制的资产和账户进行转账操作。其中,转账操作的金额不超过攻击者拥有的资产数目;转账的发起方是其能够控制的账户,包括攻击者及其盟友创建的账户,以及愿意提供资产中转服务的服务商账户等等。攻击者能够获得利益一般由其知晓私钥的账户的评分决定。如果有多个这样的账户,一种简单的情况是,攻击者的利益正比于这些账户的得分总和。当然,先前提到的服务商账户的私钥不受攻击者掌控。

本节的分析基于上述的行动空间和简单情况下对攻击者利益的定义。首先,我们讨论针对单个账户的分数提高上限;然后,我们分析攻击者多个受控账户的分数提升上限;最后,引入合谋行为,讨论多个攻击者联合操纵的结果。

%操纵有很多种,一定的简化考虑

\subsection{提升单个账户的分数}
%增加自己内部的交易量}
为了提升单个账户的分数,根据式\ref{eq:rank-param},分数与账户上的资产数目以及出入度指标正相关。账户上的资产数目,即$\beta$存在上限,即不高于攻击者的资产数目。而出入度指标$\gamma$反映了交易量的大小,这意味着攻击者需要尽可能增加单个受控账户的交易量。增加交易量分为两个方面:增加入度和增加出度。增加出入度需要两个账户参与,除了需要提升分数的账户之外,另一账户有两种情况:受控账户和非受控账户。如果是非受控账户,增加出入度意味着与其他人进行交易,此类情况将在\refsec{sec:coalition}讨论,本小节不考虑这种情况;或者攻击者可以无条件地向陌生人账户转移资产以增加出度,此类行为的代价较大,亦不予考虑。因此攻击者的行为主要是增加自己所控制的账户之间的交易。由于攻击者所控制的账户的资产有限,而评分的时间长度也有限,因此该账户的出入度之和具有上限,并由攻击者的资产数目决定。考虑一种情况,攻击者设法将所拥有的所有资产转入该账户,并随后全部转出,由于存在去环算法,攻击者的资产无法在当前时间段内再次转入,因此这是交易量最大的策略。此时的出入度之和为$\gamma = 2 \beta$。此时的分数为$\mathcal{C} =  \frac{2 \beta ^2}{ (1+e^{a + b \times{} \beta}) (1+e^{c + 2 d \times{} \beta})}$。假如攻击者采用了线下交易等方式,将资产完全平移到其他账户并再次转入目标账户,出入度之和的上限是转移次数乘以资产数目,由于评分时间段是有限的,转移次数的上限是常数,因此$\gamma$的上限是$const \times \beta$,分数的上限是$\mathcal{C} =  \frac{const \beta ^2}{ (1+e^{a + b \times{} \beta}) (1+e^{c + const d \times{} \beta})}$。

\subsection{提升多个账户的分数(女巫攻击)}
女巫攻击(Sybil Attack)是指攻击者通过创建大量的假名标识来破坏对等网络的信誉系统,使用其获得虚假的高重要性评分。~\cite{quercia2010sybil}。

two-loops attack...


对等网络上的实体是能够访问本地资源的一块软件。 实体通过呈现身份在对等网络上通告自身 。 多于一个标识可以对应于单个实体。
换句话说,身份到实体的映射是多对一的。 对等网络中的实体为了冗余,资源共享,可靠性和完整性而使用多个标识。
在对等网络中,身份用作抽象,使得远程实体可以知道身份而不必知道身份与本地实体的对应关系。
默认情况下,通常假定每个不同的标识对应于不同的本地实体。 实际上,许多身份可以对应于相同的本地实体。
对手可以向对等网络呈现多个身份,以便出现并充当多个不同的节点。 因此,对手可能能够获得对网络的不成比例的控制水平,例如通过影响投票结果。
\subsection{联合操纵 \label{sec:coalition}}
