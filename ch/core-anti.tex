% !TEX root = main.tex

\section{Nebulas Rank Core 如何抵抗操纵?}
操纵有很多种,一定的简化考虑

\subsection{增加自己内部的交易量}
\subsection{女巫攻击}
女巫攻击(Sybil Attack)是指攻击者通过创建大量的假名标识来破坏对等网络的信誉系统,使用其获得虚假的高重要性评分。~\cite{quercia2010sybil}。

two-loops attack...


对等网络上的实体是能够访问本地资源的一块软件。 实体通过呈现身份在对等网络上通告自身 。 多于一个标识可以对应于单个实体。
换句话说,身份到实体的映射是多对一的。 对等网络中的实体为了冗余,资源共享,可靠性和完整性而使用多个标识。
在对等网络中,身份用作抽象,使得远程实体可以知道身份而不必知道身份与本地实体的对应关系。
默认情况下,通常假定每个不同的标识对应于不同的本地实体。 实际上,许多身份可以对应于相同的本地实体。
对手可以向对等网络呈现多个身份,以便出现并充当多个不同的节点。 因此,对手可能能够获得对网络的不成比例的控制水平,例如通过影响投票结果。
\subsection{联合操纵}
