% !TEX root = main.tex

\section{区块链经济模型}
区块链之上的加密数字货币无疑正在成为一种新的经济体~\cite{neweco},这一经济体将包含越来越多的行为及个体,整个系统也逐渐变得更加复杂,由最初的、单纯的数字货币系统,正在逐步成为智能合约、甚至应用的基础平台。

尽管区块链之上的加密数字货币是一种新型的经济体,我们认为,其仍然遵循必然的经济规律。也就是说,数字货币对应的经济体的变化仍然遵循经济学中的客观规律,
使用传统的、严谨的经济学方法对加密数字货币进行研究依然是有效的。

我们希望Nebulas能够成为一个高效的经济体,既,Nebulas能够被大家使用,用于各种场景下的交易、并支撑各种智能合约及应用。
因此,我们需要深入的理解一个经济体是如何变得高效的,以及如何定量的描述这些指标。

在传统的经济学中,存在很多指标用于描述经济系统,例如``流动性'',``流通性'',``货币流动性''等。
我们认为使用``货币流通速度''是一个便于使用的指标,Selden~\cite{selden}指出,货币流通速度是一个时期中货币的流量与该时期
中货币平均存量的比率。也就是说xxxx。


{\color{gray} 在技术白皮书中,我们使用了``流动性''一词,然而,``流动性''一词缺乏严格的定义,即使在经济学中,这一涵义也是十分广泛的。
例如,在《新帕尔格雷夫金融学辞典》中,对流
动性(liquidity) 解释的专门词条包括了完全不同的三个方面。兰德尔·克罗兹勒~\cite{randall}指出,在过去六个月里,有2795篇
独立的文章谈到了流动性,但流动性是什么含义,大概有2795种不同的说法。}

\subsection{市场行为模型}
在此我们建立一个简单的市场行为模型来帮助理解“流通性”的重要性,首先我们给出涉及到参数的定义。
\begin{itemize}
\item{T:市场中所有参与者使用星云币进行支付的总交易量,用每秒钟发生的交易量来计量,这里采用美元作为单位方便后面统一价值;}
\item{D:星云币用于支付时,暂时退出流通体系的时间跨度,单位为秒;}
\item{S:市场上流通的星云币总量,其值等于星云币总量减去人们打算长期持有的星云币数量;}
\item{P:星云币价格(对美元);}
\end{itemize}
在供应侧,我们可以计算出每秒有$S/D$个星云币重新进入流通体系,而在需求侧,每秒钟所需用于支付交易的星云币数量为交易规模T(美元)乘以1美元对应$1/P$个星云币,即$T/P$。当达到供需平衡时,我们可以得到以下公式:
\begin{align}
S/D=T/P
\end{align}

从中推出星云币价格:
\begin{align}
P=\frac{TD}{S}
\end{align}

通常情况下D稳定(交易过程中星云币退出市场的持续时间不变),假设S不变或者变化缓慢(考虑到目前星云币增发比例暂定为4\%,市场流通总量较为稳定),那么星云币价格P和交易规模T成正比,即当交易需求增加,对应星云币价格也会增加。尽管上述模型较为简单,并未考虑投资者预期等复杂因素,但根据此模型已经能够分析得出货币价值与市场交易需求正相关的结论,这也与我们普遍认知一致。

需要注意的是,当用户长期持有星云币比例增加时,市场流通星云币则会减少,从而也会增加星云币价格。此时则反映出星云币的投机资产特性,而削弱了其交易特性。这也是目前比特币等通缩货币所面临的主要问题,经济学称之为“大国会山托儿合作社危机”(The Great Capitol Hill Baby Sitting Co-op Crisis)~\cite{hens2007great},在此暂时不做详细讨论。

\subsection{交易激励}

我们认为在经济系统中的排名,意味着一种激励的方向,这个方向应该与经济系统增长的方向保持一致。为了保证或者进一步提升经济系统的货币的价值,根据上述模型,需要增加市场的交易规模。对应地,在Nebulas经济系统中,需要激励用户的交易行为。

我们将贡献度(Salience)作为衡量用户重要性指标之一,更进一步的,我们认为,经济系统中的排名应该反应其对经济系统的贡献。

%更加形式化的,f(a+b) > f(a) + f(b) 及 f(a+b) < f(a) + f(b)
