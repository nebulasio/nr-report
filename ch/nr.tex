\section{星云指数}
如前所述,星云指数反映了用户对经济产量的贡献,为了量化的计算星云指数,我们首先需要对系统给出形式化的模型的描述,
然后,我们给出了一般意义上星云指数的计算方法。


\subsection{计算函数}
考虑到不同的使用场景及不同的性质,星云指数的计算是十分复杂的,因此,具体的星云指数的计算是依赖于场景的,然而,我们可以总结出一般意义上的星云指数计算函数的性质。

我们记星云指数的计算函数为\(f(x)\),其中\(x\)
为星云指数需要参考的因素,可以为持有的余额、币龄或账户的出入度。为了操纵星云指数,
攻击者可以进行任意的操作,包括创建足够多的账户、进行账户之间的转账等,在诸多攻击方式中,唯一确定的事实是,
{\color{red} 用户需要将原本属于一个账户的资金拆分为多份,并转移到其他账户中},因此,为了抵抗操纵,
需要保证用户在拆分资金后,其星云指数会降低,即:

\begin{align}
f(a + b) > f(a) + f(b).\quad a>0, b>0
\end{align}

需要注意的是,上式可能会产生另外一种可能的操纵幸运指数的方式,即多个用户通过将账户中的余额集中
到同一个账户中,以获得更高的收益,因此,需要满足

\begin{align}
\lim\limits_{a \to \infty, b\to \infty} f(a+b) = f(a) + f(b).\quad a>0, b>0
\end{align}
满足上述两个性质的函数有很多,在此,我们仅给出一个满足上述性质的函数
\begin{align}
f(x) = x/(1 + e^{a + b*x})
\end{align}
\noindent 该函数的图形如\reffig{fig-nr}所示。可以证明,当一个账户中的资金被拆分到多个账户中后,
多个账户的星云指数之和是减少的,更进一步的,当拆分的账户越多,则这些账户的星云指数之和会越小。

\begin{figure}
\centering
\begin{tikzpicture}[
    declare function={func(\x,\mu) = (\x / (1 + exp(\mu-\x)));},
    declare function={linefunc(\x) = \x;}
]
\begin{axis}[
    axis lines=left,
    enlargelimits=upper,
ticks=none,axis x line=bottom,axis y line=left,xlabel={Nebulas Rank Factor},
  ylabel={星云指数},
      legend pos=north west
]
\addplot [dotted, domain=0:10, blue] {linefunc(x)};
\addplot [smooth, domain=0:10, red] {func(x,3)};
\addlegendentry{$f(x)=x$}
\addlegendentry{$f(x)=x/(1 + e^{a + b*x})$}
\end{axis}
\end{tikzpicture}
\caption{星云指数计算函数曲线}
\label{fig-nr}
\end{figure}

Nebulas Rank的计算需要满足上述两个性质,我们给出一个满足上述性质的函数





NR试图衡量经济流通性

然而这很困难,大牛表示,只是从图来做这个事情,是搞不定的

更进一步的,各种复杂的场景,需要的Rank非常多样化

因此,我们为白皮书中的东西,定义为NR Core,更广义的使用场景,定义为NR Extension
