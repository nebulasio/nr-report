\section{Nebulas Rank Core}

Nebulas Rank Core的设计目标是量化的衡量用户对整个经济体的贡献。

精确的计算这一指标是十分困难的,因此,我们使用了一个近似算法。
在该近似算法中,我们考虑了两个重要的因素,即账户持有的币龄及账户在交易网络中的位置信息。
我们认为,仅仅考虑币龄会使Nebulas Rank Core蜕化为PoS,而仅仅考虑账户在交易网络中的位置,则使得
Nebulas Rank Core无法抵抗操纵。


\subsection{问题描述} \label{subsec:txg}
我们选择使用链上的交易记录作为\textbf{Nebulas Rank}算法的数据来源。因为相比于现实世界,区块链世界的“行踪”更为清晰和可信:
链上的交易数据忠实地记录了用户之间的每笔转账、以及每次对“智能合约”的调用情况。然而根据交易数据设计算法并非易事。
因为相比于现实世界,区块链世界的交易具有天然的匿名性,同时数据量规模更为庞大。因此我们试图为\textbf{Nebulas Rank Core}刻画如下的性质:

我们用图的形式表现交易记录。在交易图的基本定义中,每个节点代表一个实体,
每条边代表实体之间的转账行为~\cite{Tschorsch2015}。交易图表现了转账行为导致资产流动的性质。

对于任何一个时刻$t_0$,我们记录时间 $[t_0\ −\ T,\ t_0]$ 内,
所有区块包含的个人用户之间的\textbf{有效}交易记录。其中,每笔交易记录可以表示成一个4-元组$(s,r,a,t)$,
$s$ 是转出地址,$r$ 是转入地址,$a$ 是交易金额,$t$ 是这笔交易的区块时间。
并且,我们定义任何一个交易是\textbf{有效}的,当且仅当$a > 0$和$s \neq r$。
因而,$[t_0\ −\ T, t_0]$ 内所有的有效交易记录可以表示成一个4-元组的集合:

\begin{align}
\Theta(t_0) = \{(s, r, a, t)\ |\ t_0 - T \le t \le t_0\ \land \ a > 0 \land s \neq r \}
\end{align}

问题:求每个账户的Nebulas Rank Core的值。

\subsection{问题解法}


基于$\Theta(t_0)$,我们可以构造有向加权简单图 $G = (V, E, W)$,其中,$V$、$E$和$W$ 分别表示节点集合、有向边集合和边的权值集合。$V$ 中的每个节点都代表一个账户的地址,$E$ 中的每条有向边代表两个用户之间的交易关系。我们定义以下公式合并对应交易的最高$K$ 个金额作为权值$w_e$ :
\begin{align}\label{formula:edgeweight}
w_e = \sum_{i=1}^K a_i,\ s.t.\ a_i \in A_e
\end{align}
其中$A_e$ 是一个由所有在 $\Theta(t_0)$ 出现的从 $s$ 到 $r$ 的交易的金额 $a$ 组成的有序集合:
\begin{align}
A_e = \{a_i\ |\ e = (s,r) \land (s, r, a_i, t) \in \Theta(t_0) \land (a_i \ge a_j, \forall i \le j) \}
\end{align}

另外,记 $N = |V|$,$M = |E|$,其中 $|.|$ 表示集合的范数。因而,总共有$N$ 个地址和$M$ 条有效交易。在下文中,为了表述便利,所有节点都以一个$1$到$N$ 之间的整数来表示。

接着,对每个节点,根据其在$[t_0\ −\ T,\ t_0]$的转入转出记录,
计算它的“币龄”,记作$C_v$;根据转入转出的金额总量,


最后,取整个交易图的最大弱连通分支,将分支之外的节点删除。被删除的节点不参与后面的排名,默认赋予最低的重要性分数。



NRC的问题定义

NRC的解决思路

几种可能的解决方案
