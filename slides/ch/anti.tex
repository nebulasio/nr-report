\section{抵抗操纵}
\begin{frame}
“操纵”指的是攻击者作出特定行动以获得最高的利益。攻击者的行动空间是利用自己和盟友所能控制的资产和账户进行转账操作。

%其中,转账操作的金额不超过攻击者拥有的资产数目;转账的发起方是其能够控制的账户,包括攻击者及其盟友创建的账户,以及愿意提供资产中转服务的服务商账户等等。攻击者能够获得利益一般由其知晓私钥的账户的评分决定。如果有多个这样的账户,

一种简单的情况是,攻击者的利益正比于这些账户的得分总和。

%本节的分析基于上述的行动空间和简单情况下对攻击者利益的定义。首先,我们讨论针对单个账户的分数提高上限;然后,我们分析攻击者多个受控账户的分数提升上限;最后,引入合谋行为,讨论多个攻击者联合操纵的结果。
\begin{itemize}
\item 提升单个账户的分数
\item 提升多个账户的分数(女巫攻击)
\item 联合操纵
\end{itemize}

\end{frame}

\begin{frame}
\frametitle{提升单个账户的分数}
攻击者需要尽可能增加单个受控账户的交易量。%增加交易量分为两个方面:增加入度和增加出度。增加出入度需要两个账户参与,除了需要提升分数的账户之外,另一账户有两种情况:受控账户和非受控账户。


考虑与受控账户交易的情况,由于我们提出的计算函数,攻击者拆分资产交易的收益会被降低。因此攻击者会采取交易量最大策略,即设法将所拥有的所有资产转入该账户,并随后全部转出,由于存在去环算法,攻击者的资产无法在当前时间段内再次转入。此时的出入度之和为$\gamma = 2 \beta_0$。

此时的分数为$\mathcal{C} =  \frac{2 \beta_0 ^2}{ (1+e^{a + b \cdot \beta_0}) (1+e^{c + 2 d \cdot \beta_0})}$。
\end{frame}


\begin{frame}
\frametitle{提升多个账户的分数(女巫攻击)}
女巫攻击(Sybil Attack)是指攻击者通过创建大量的假名标识来破坏对等网络的信誉系统,使用其获得虚假的高重要性评分。

攻击者所拥有的所有账户分数之和的上限是:
\begin{align}
\mathcal{C} = N \cdot \frac{K \frac{\beta_0 ^2}{N}}{ (1+e^{a + b \cdot \frac{\beta_0}{N} }) (1+e^{c + K \cdot d \cdot \beta_0})} = \frac{K \beta_0 ^2 }{ (1+e^{a + b \cdot \frac{\beta_0}{N} }) (1+e^{c + K \cdot d \cdot \beta_0})} 
\end{align}
\end{frame}


\begin{frame}
\frametitle{联合操纵}
联合操纵的结果和一个攻击者拥有了原先两个攻击者的资产数目的情况相同,因此可以通过分析单个攻击者资产增加的后果来分析合谋的结果。
\end{frame}