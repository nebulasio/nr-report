\section{核心星云指数的实现}
\begin{frame}\frametitle{是否上链?}
	我们认为并不需要将星云指数上链,这出于以下两个方面的考虑:
\begin{itemize}
\item 链上不适合存储如此大量的数据,即使对于IPFS,Genaro之类的以数据存储为目标场景的公链,
也不适合周期性的存储所有账户的核心星云指数;
\item 对于核心星云指数的计算会影响出块速度,核心星云指数的计算复杂度较高,如果将计算结果上链,
将很大程度上影响出块及验证速度,导致整个系统的TPS降低。
\end{itemize}
\noindent 综上,我们认为每个节点可以根据需要自行计算核心星云指数。
\end{frame}

\begin{frame}\frametitle{核心星云指数的更新}
我们会更新区块结构,新的区块结构中将包含核心星云指数的算法及参数(以LLVM IR形式),
星云虚拟机(NVM)作为算法的执行引擎,从区块中获得核心星云指数的算法及参数,并执行算法,在节点内获得账户的核心星云指数。

在算法或参数需要更新时,我们将和社区一起在新的区块中包含最新的算法及参数,
从而保证整个更新过程的及时性及平滑性,亦避免了可能到来的分叉。
\end{frame}

\section{扩展星云指数}

\begin{frame}\frametitle{针对智能合约的扩展星云指数}
对智能合约的排序,基于两个事实,即账户地址对智能合约的调用,及智能合约之间的调用。我们首先将账户地址对智能合约的调用看作账户地址向智能合约分摊
自己对经济总量的贡献,从而使得每个智能合约有了初始的分值;然后将智能合约之间的调用看作有向无环图,使用Page Rank对每个智能合约计算最终的星云
指数。
\end{frame}

\begin{frame}\frametitle{多维的扩展星云指数}
一些应用需要多个维度的数据,对链上数据的相关性进行计算,例如基于区块链的广告系统,需要在多个维度对需要投放的广告及用户进行
相关性计算。在这种场景下,扩展星云指数是多维的,即,表示为一个向量,核心星云指数作为其中的一个维度。

对于多维的扩展星云指数其他维度,我们认为依赖于具体的应用场景,可以参考本文中对于核心星云指数的计算方法。对于多维的扩展星云指数的使用,同样依赖于具体
的应用场景。
\end{frame}

\section{未来工作}
\begin{frame}\frametitle{跨链的星云指数}
可以预见到,跨链的数据转移将成为不可避免的应用需求。例如,跨链的数据交互、数字资产转移,
这需要对不同链上的价值作出衡量;开发者将DApp从一个链迁移到另一个链,该DApp在一个链上星云指数如何在另一个链上
\end{frame}

\begin{frame}\frametitle{更多的对经济总量贡献的指标}
星云指数的基础是经济总量的贡献,然而,区块链的发展离不开社区,社区的增长对
经济总量的贡献是不可忽视的,如何衡量个体或组织对社区增长的贡献,并将其反映到星云指数中,具有不可忽视的现实意义。
\end{frame}